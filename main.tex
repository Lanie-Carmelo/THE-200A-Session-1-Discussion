% APA 7 Student Paper Template
% Author: Lanie Molinar Carmelo
% Usage: Fill in metadata, write content, and build with 'make pdf'
% NOTE: This template uses biblatex with the biber backend.
%       If building manually, run:
%         pdflatex main.tex
%         biber main
%         pdflatex main.tex
%         pdflatex main.tex

\documentclass[stu,12pt,floatsintext]{apa7}

% Language and citation setup
\usepackage[american]{babel}
\usepackage{csquotes}
\usepackage[style=apa,sortcites=true,sorting=nyt,backend=biber]{biblatex}
\DeclareLanguageMapping{american}{american-apa}
\addbibresource{references.bib}

% Font and encoding
\usepackage[T1]{fontenc}
\usepackage{newtxtext,newtxmath}  % Modern Times-like font with math support

% Document metadata
\title{Session 1 Discussion}
\author{Lanie Molinar}
\authorsaffiliations{Colorado Christian University}
\duedate{June 11, 2025}
\course{Introduction to Systematic Theology (THE-200A)}
\professor{<Cari Nimeth}

\begin{document}

\maketitle
\thispagestyle{plain}
\pagestyle{plain}

\section{What is your present concept of God? Explain.}
I think of God as a loving Father, someone who wants the best for me and understands me even when I feel like no one else does. I do not have a good relationship with my earthly father, as he has never been there for me and seems uncomfortable around me, perhaps due to my disabilities, so this concept of God has always stood out to me. I often feel lonely and like no one understands what I go through as a blind, autistic woman with a lengthy list of health conditions, and it can help to remember that Jesus Christ understands what I go through when I feel this way. This understanding of God as a loving Father who knows and cares for me is consistent with what \textcite{erickson_introducing_2015} describes as God’s self-revelation through Scripture and through Jesus Christ.

\printbibliography

\end{document}